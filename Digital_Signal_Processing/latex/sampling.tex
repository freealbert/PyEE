\documentclass{article}
\usepackage[colorlinks,linkcolor=red,anchorcolor=blue,citecolor=green]{hyperref}
\usepackage{tikz}
% \usepackage{KOMA}
\usetikzlibrary{shapes,snakes}
\usepackage{graphicx}
\usepackage{fontspec}
\usepackage{caption}
\XeTeXlinebreaklocale "zh" %针对中文进行断行
% \setmainfont{Microsoft YaHei} %设置默认的字体
% \setmainfont{FZShouJinShu-S10}
% \setmainfont{Hiragino Sans GB}
\setmainfont{Hiragino Sans GB}
\begin{document}
\tikzstyle{mybox} = [draw=blue, fill=green!20, very thick ,rectangle, rounded corners, inner sep=10pt, inner ysep=20pt]
\tikzstyle{fancytitle} = [fill=blue,text=white]
{\setmainfont{FZShouJinShu-S10}
\title{{\setmainfont{FZShouJinShu-S10}\bf{PyEE之数字信号处理篇}}}
\author{freealbert\\\href{http://dspandlinux.com}{http://dspandlinux.com}}
\maketitle
}
\section{Introduction}

Hello, 大家好,PyEE新番上市!\\

作为PyEE走入课堂的一次尝试,考虑到很多同学尚未接触过EE专业必学必会的Matlab,为了降低难度,DSP部分将给出所有仿真程序的Python代码作为参考,教师可以指导学生使用Matlab完成实验。Python和Matlab一样,语法较为清晰易懂,科学计算的函数库完备且很多函数名和用法相同,甚至接口风格一样(这一直是吐槽点),作为参考再好不过,同时两者语法又不尽相同,可以确保在给学生充分提示的同时,仍旧需要查阅资料抑或help一下以学个究竟。

本次实验分为两部分,简而言之,就是上采样和下采样。
废话不多说,用愉快的心情来享受这一切吧。
\section{Up Sampling}
上采样,又名信号插值(interpolation),是一个增大采样率来增加数据的过程。假设,有限长的离散时间信号$x(n),n=0,1,\cdots,N-1$,需要将采样频率$f_s$提升$L$倍,上采样后的离散序列可以表示为
\begin{equation}
	x_U(n)=\left\{
	\begin{array}{ll}
		x(n/L), & n=0,\pm L,\pm 2L,\cdots\\
		0, & else
	\end{array}\right.
\end{equation}
\begin{tikzpicture}
\node [mybox] (Qustion_1) {经过升采样后,信号的频域波形会有何变化?};
\node [fancytitle] at (Qustion_1.north west) {\bf {Question 1}};
\end{tikzpicture}
\subsection{时域和频域图像绘制 I}
给定信号$x(t)=\sin(100\pi t)+\sin(200\pi t)$,采样频率$f_s=1000Hz$,采样时间$\tau=0.05s$
请使用Matlab画出如下图所示的$x(t)$的时域波形和频谱图,并保存图片。\\
{\small \setmainfont{AR PL UKai CN}{
Tips: 对于如何画图有疑问时,最好的方法就是在Matlab的Command Window中输入 help yourFunction ,yourFunction指你需要查询的函数,你大体需要以下函数 stem,plot,fft,fftshift}}
\subsection{时域和频域图像绘制 II}
令$\tau$取不同值,如$0.05,0.2,\cdots$,画出$x(t)$的时域波形和频谱图。\\
\begin{tikzpicture}
\node [mybox] (Qustion_2) {观察频域波形有何不同,试着解释下原因。};
\node [fancytitle] at (Qustion_2.north west) {\bf {Question 2}};
\end{tikzpicture}\\
{\small \setmainfont{AR PL UKai CN}{
Tips: 当绘出全部时间的图形时,图像会显得过于密集反而不利于观测,这时可像Figure 1一样只绘出一小段时间片的图像}}
\subsection{插值}
令插值倍数$L=4$画出插值后信号$x'(t)$的时域和频域图。验证下是否与你在Question 2中所想的一样。
\subsection{插值后的处理}
\begin{enumerate}
\item 观察插值后的时域和频域图像与插值前有何差异。
\item 是否能够容忍这种差异?
\item 如不能容忍,那该如何去除?
\end{enumerate}
我才不会告诉你要低通滤波呢。
\end{document}